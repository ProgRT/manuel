\chapter{Introduction}

Le VDR-4 est un appareil de ventilation à haute fréquence conçu à la fin des années 1980 par l'inventeur américain Forest Morton Bird. 
Il a été conçu en tant qu'appareil de ventilation universel, capable de ventiler adéquatement n’importe quel poumon humain, sain ou gravement malade, de la clientèle néonatale à la clientèle adulte. 
Son développement a été motivé par le constat que la ventilation mécanique en pression positive conventionnelle (c’est-à-dire basée sur un volume courant et une fréquence physiologiques) était peu adaptée à la ventilation d'un patient présentant une pathologie pulmonaire inhomogène (MPOC, pneumonie, brûlure d’inhalation, SDRA, etc.).

Outre le type singulier de ventilation qu’il délivre, le VDR-4 se distingue aussi par son fonctionnement entièrement pneumatique. Ceci lui confère l'avantage d’être entièrement indépendant de toute source d'alimentation électrique. En contrepartie, l'appareil a des capacités de monitorage très limitées.

\section{Vocabulaire}

\begin{description}
	\item[Convection:] Déplacement d’un volume de gaz. Lors de la ventilation <<conventionnelle>>, les échanges gazeux entre les alvéoles et le circuit du ventilateur se font par convection. On peut donc parler de ventilation convective.
	\item [Diffusion:] Déplacement des molécules d’un gaz à l’intérieur d’un mélange gazeux. Les molécules d’un gaz diffusent en suivant leur gradient de concentration.
	\item [Hertz:] Unité de mesure de fréquence correspondant à un cycle par seconde ou soixante cycles par minute.
	\item [Iatrogène:] Causé par la thérapie.
	\item [Percussion:] Bref jet de gaz à haute vélocité.
	\item [Pression motrice:] Dans un contexte de ventilation par VDR-4, on désigne pression motrice la différence entre la pression moyenne à l’inspiration et la pression moyenne à l’expiration.
	\item [Pression partielle:] Pression exercée par les molécules d’un gaz à l’intérieur d’un mélange gazeux.
\end{description}

\section{Notions de ventilation à haute fréquence}

Ce qui caractérise la ventilation à haute fréquence est l'administration de volumes courants inférieurs au volume de l'espace mort anatomique du patient. Les échanges gazeux entre les alvéoles et le circuit de ventilation s'y font selon un ensemble de mécanismes différents. À ce jour, l'influence respective de chacun de ces mécanismes reste encore à élucider.

\subsection{Oxygénation lors de la ventilation à haute fréquence}

Les facteurs influençant l'oxygénation lors de la ventilation à haute fréquence sont, à toute fin pratique, les mêmes que pour la ventilation convective.

Dans l'absolu, l'oxygénation du sang est proportionnelle à la pression partielle d'oxygène dans les alvéoles. Les trois principales variables influençant cette pression partielle sont:
la concentration d'oxygène dans l'air insufflé,
la pression alvéolaire moyenne,
la concentration alvéolaire de gaz carbonique.

\subsection{Relation fréquence-volume-ventilation}

Ce qui limite les volumes courants en ventilation à haute fréquence est le peu de temps disponible pour chaque cycle respiratoire. Ce temps est d'autant plus court que la fréquence est élevée. 

$$ {T_{cycle}}_{(s)} = \frac{60}{Freq._{(cycle/min)}}$$

En conséquence, une diminution de la fréquence entrainera une augmentation du volume courant en laissant plus de temps à la pression pour s'équilibrer entre le circuit et les alvéoles. Inversement, une augmentation de la fréquence entrainera une diminution du volume courant.

Ainsi, en ventilation à haute fréquence, une diminution de la fréquence favorise une plus grande élimination du $CO_2$.

Il a été démontré que la fréquence est un paramètre très important pour l'élimination du $CO_2$ en ventilation à haute fréquence\cite{Pillow2005}.

\section{Particularité du VDR-4}
Le VDR-4 se distingue des autres appareils de ventilation à haute fréquence par l’alternance (à basse fréquence) entre deux (voire même trois) amplitudes de percussion. Il en résulte une alternance entre deux pressions moyennes. Les échanges gazeux lors de ce type de ventilation seront, par conséquent, à la fois le résultat du déplacement de volumes d’air (convection) et de l’accélération de la diffusion propre à la ventilation à haute fréquence.

\begin{figure}
	\tikzsetnextfilename{fig-lfhf}

\pgfplotsset{
	lfhf/.style={
		height = 0.42 \textheight,
		enlarge y limits = {value=0.9, upper},
		enlarge x limits = false
		}
}

\tikzset{
	zoomline/.style={
		opacity=1,
		dotted
	},
	plage/.style={
		<->
	}
}

\def\zstart{7.2}
\def\zend{7.8}
\newcommand{\istart}{2}
\newcommand{\tic}{2}
\newcommand{\pstart}{7.365}
\newcommand{\tip}{0.059}

\begin{tikzpicture}
	\begin{groupplot}[
			group style={
				group size=1 by 2,
				y descriptions at=edge left,
				xlabels at=edge bottom
				},
				ylabel=Pression (hPa),
				xlabel=Temps (s),
				max space between ticks=40,
			]
		\nextgroupplot[lfhf, width=\textwidth, height=5cm]

		\addplot []table[x=time, y=Pao] {dat/simvent1.dat};


		\coordinate (PSO) at (axis cs:\zstart,0);
		\coordinate (PSE) at (axis cs:\zend,0);
		\coordinate (PNO) at (axis cs:\zstart,\pgfkeysvalueof{/pgfplots/ymax});
		\coordinate (PNE) at (axis cs:\zend,\pgfkeysvalueof{/pgfplots/ymax});

		\draw [plage](axis cs:\istart,45) -- (axis cs:\istart + \tic, 45) node[midway, above] {Inspi.};
		\draw [plage](axis cs:\istart + \tic,45) -- (axis cs:\istart + 2*\tic, 45) node[midway, above] {Expi.};

		\draw [dashed] 
		(axis cs: \istart,\pgfkeysvalueof{/pgfplots/ymax}) -- (axis cs:\istart,0)
	 	(axis cs: \istart + \tic,\pgfkeysvalueof{/pgfplots/ymax}) -- (axis cs:\istart + \tic,0)
		(axis cs: \istart + 2 *\tic,\pgfkeysvalueof{/pgfplots/ymax}) -- (axis cs:\istart + 2*\tic,0);

		%\fill [opacity=0.15] (PSO) rectangle (PNE);
		\draw [zoomline] (PSO) rectangle (PNE);


		\nextgroupplot[lfhf,
				max space between ticks=80,
				width=0.75\textwidth,
				height=5cm,
				axis background/.style={fill=gray!15, opacity=0.8},
				]
		\addplot [restrict x to domain=\zstart:\zend]table[x=time, y=Pao] {dat/simvent1.dat};

		\coordinate (ZNO) at (axis cs:\zstart,\pgfkeysvalueof{/pgfplots/ymax});
		\coordinate (ZNE) at (axis cs:\zend,\pgfkeysvalueof{/pgfplots/ymax});
		\coordinate (ZSO) at (axis cs:\zstart,\pgfkeysvalueof{/pgfplots/ymin});
		\coordinate (ZSE) at (axis cs:\zend,\pgfkeysvalueof{/pgfplots/ymin});


		\draw [dashed] 
		(axis cs: \pstart,\pgfkeysvalueof{/pgfplots/ymax}) -- (axis cs:\pstart,0)
		(axis cs: \pstart + \tip,\pgfkeysvalueof{/pgfplots/ymax}) -- (axis cs:\pstart + \tip,0)
		(axis cs: \pstart + 2 *\tip,\pgfkeysvalueof{/pgfplots/ymax}) -- (axis cs:\pstart + 2*\tip,0);
		\draw [plage] (axis cs:\pstart,45) -- (axis cs:\pstart + \tip, 45) node[midway, above] {Ins.};
		\draw [plage](axis cs:\pstart + \tip,45) -- (axis cs:\pstart + 2*\tip, 45) node[midway, above] {Exp.};

	\end{groupplot}

	\begin{scope}[on background layer]
		%\fill [opacity=0.03](ZNO) -- (PNO) -- (PNE) -- (ZNE) -- (ZNO);
		%\fill [opacity=0.03](ZSO) -- (PSO) -- (PNO) -- (PNE) -- (PSE) -- (ZSE) -- (ZSO);
		\fill [opacity=0.1](PSO) -- (PNO) -- (PNE) -- (PSE);
		%\fill [opacity=0.05](ZNE) -- (PNE) -- (PSE) -- (ZSE) -- (ZNE);
		\draw [zoomline](ZNO) -- (PNO) (PNE) -- (ZNE) ;
		\draw [zoomline](ZSO) -- (PSO) (PSE) -- (ZSE) ;
	\end{scope}

\end{tikzpicture}

	\caption{L’alternance entre deux amplitudes de percussions donne une apparence typique au tracé de la pression à l'ouverture des voies aériennes lors de la ventilation avec un VDR-4. Les phases inspiratoires et expiratoires à basse fréquence (courbe du haut) sont composées d’une succession d’inspirations et d’expirations à haute fréquence (courbe du bas).}
\end{figure}
