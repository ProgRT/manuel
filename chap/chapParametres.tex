\def\knobshow#1{%
\marginpar{%
	\centering
	\begin{tikzpicture}
\pic [pic text=#1, scale=1.5] {aknob};
\end{tikzpicture}%
}}

\chapter{Paramètres de ventilation}

Le réglage des paramètres de ventilation se fait en ajustant l’ouverture de
valves sur le module de contrôle. Les valves du module de contrôle sont
identifiées par le principal paramètre visé par le réglage. Cependant, le
réglage de l’ouverture d’une valve entraine presque toujours la modification
d’au moins deux paramètres. Un code de couleurs identifie les valves en
fonction du type de paramètre visé par son réglage.  

\section{Paramètres d’amplitude}

Une amplitude de percussion différente peut être réglée pour chacune des trois
phases du cycle de convection (basse fréquence).  Ces trois réglages sont
identifiés par la couleur verte sur le module de contrôle. 

\subsection{Amplitude des percussions à l’inspiration (phase haute)}

Il s’agit du paramètre de base à partir duquel sont réglés les deux autres
paramètres d’amplitude. Cela signifie qu’une modification de ce paramètre
entrainera une modification dans la même direction des deux autres amplitudes.
Cette amplitude est réglée au moyen de la valve identifiée Debit pulse.
\knobshow{D}

\subsection{Amplitude des percussions à l’expiration (phase basse)}

L’amplitude des percussions pendant l’expiration convective est réglée par
comparaison à celle pendant l’inspiration convective. Cela signifie qu’une
modification de l’amplitude à l’inspiration entrainera une modification de
l’amplitude à l’expiration. Par contre, l’amplitude à l’inspiration ne sera pas
affectée par une modification de celle à l’expiration. Cette amplitude est
réglée au moyen de la valve identifiée CPAP oscillante.
\knobshow{O}

\subsection{Amplitude de percussion augmentée (troisième phase)}

Lorsqu’elle est activée, la troisième phase commence 0,8 seconde après le début
de l’inspiration convective (voir Figure 10). Il en résulte une inspiration en
deux temps. Cette amplitude est réglée au moyen de la valve identifiée pression
de convection. Ce paramètre n’est pas utilisé dans le protocole clinique en
vigueur au CHUM.
\knobshow{C}

\begin{figure}
\caption{Tracé de la pression à l'ouverture des voies aériennes. On peut
observer une augmentation de la pression 0.8 secondes après le début de
l’inspiration.}
\end{figure}

\begin{table}
	\begin{tabular}{l l}
		\hline
		Paramètre & Désignation sur l’appareil\\
		\hline
		Amplitude à l’inspiration (phase haute) & DEBIT PULSE\\
		Amplitude à l’expiration (phase basse) & CPAP OSCILLANTE\\
		Amplitude augmentée (troisième phase) & PRESSION DE CONVECTION\\
\hline
	\end{tabular}
	\caption{Désignation des contrôles relatifs à l'amplitude de percussion.}
\end{table}

\section{Paramètres de cyclage à haute fréquence}

Les valves contrôlant le cyclage à haute fréquence sont identifiées par la
couleur grise sur le module de contrôle.  Bien que les deux boutons permettant
de régler le cyclage à haute fréquence soient désignés «FRÉQUENCE DE
PERCUSSION» et «RATIO i:e» sur l’appareil, il s’avère que chacun de ces deux
\knobshow{i}
réglages influence la fréquence.  En fait, le bouton «FRÉQUENCE» modifie le
temps inspiratoire des percussions sans modifier le ratio i:e. Il en résulte
donc une modification de la fréquence avec un ratio i:e constant.  Quant au
bouton « RATIO i:e » il ajuste le ratio i:e sans modifier le temps
inspiratoire. Il en résulte qu’une modification du ratio i:e modifie aussi la
fréquence de percussion.
\knobshow{e}

\section{Paramètres de cyclage à basse fréquence}

Le cyclage à basse fréquence se règle en ajustant un temps inspiratoire et un
temps expiratoire.  La fréquence et le ratio inspiration : expiration
résulteront des temps inspiratoire et expiratoire réglés. Les valves contrôlant
le cyclage à basse fréquence sont identifiées par la couleur noire sur le
module de contrôle.

\section{PEP non oscillante}

La fonction PEP non oscillante (DEMAND CPAP / PEEP) est identifiée par un
bouton de couleur jaune.  Cette fonction est destinée à réduire le travail
respiratoire lors d’essai de respiration spontanée. Elle est généralement
désactivée lors de la percussion. Lorsqu’elle est activée, un débit continu est
injecté dans le phasitron. Ce débit, qui sera amplifié par le phasitron,
facilite l’inspiration et maintient une pression positive à l’expiration (en
maintenant le tube de venturi en position partiellement avancée).  

\section{Autres paramètres}

\subsection{Pression de travail}

La pression de travail est la pression à laquelle les gaz entrent dans le
circuit de logique pneumatique. Celle-ci influence à la fois l’amplitude des
percussions et les paramètres de cyclage. Chez l’adulte, on utilise
généralement la pression la plus élevée pouvant être atteinte (plus ou moins 40
\psi, selon la source d’alimentation en gaz pressurisé).


\subsection{Nébulisation}

Active ou désactive le débit destiné à actionner le nébuliseur (plus ou moins
20 l/min). Actif même lorsque la percussion est arrêtée. 

\subsection{Marche arrêt}

S’applique à la percussion seulement. Toutes les autres fonctions
(nébulisation, PEP non percussive, monitorage) demeurent actives.

\section{Interactions des paramètres}

Pour faire une règle simple, on peut dire, sans trop exagérer, que n'importe
quel paramètre peut potentiellement influencer n'importe quel autre paramètre. 

\subsection{Paramètres influençant le cyclage}

Étant donné que l’alternance entre les inspirations et les expirations des
percussions (cyclage haute fréquence)  est contrôlé par des cartouches
pneumatiques, tout paramètre influençant la pression disponible pour actionner
les cartouches peut influencer la fréquence des percussions et leur ratio i:e.
Parmi ces paramètres, on compte entre autres :

\begin{itemize}
\item La pression de travail, 
\item La \fio, 
\item Le réglage d’amplitude des percussions (DEBIT PULSE).
\end{itemize}

Le même principe s’applique au cyclage à basse fréquence (temps inspiratoire et
expiratoire convectif).

\subsection{Influence du ratio i:e des percussions sur les pressions de ventilation}

Le ratio inspiration : expiration (i:e) des percussions a une grande influence
sur les pressions de ventilation. Plus le ratio i:e est élevé, plus les
pressions serons élevées.

\section{Séquence des réglages}

\begin{figure*}
	\centering
\begin{tikzpicture}

	\pic [name=VDR, black!60, scale=0.8] {vdr};

	\begin{scope}[
		every node/.style={
			color=black,
			}
			]
	\node (1) at (VDR-e) {1};
	\node (2) at (VDR-i) {2};
	\node (3) at (VDR-F) {3};
	\node (4) at (VDR-O) {4};
	\node (5) at (VDR-I) {5};
	\node (6) at (VDR-E) {6};
	\end{scope}

	\begin{scope}[
		every node/.style={
			yshift=8mm,
			align=center,
			scale=0.5
			}
			]
	\node at (VDR-e) {RAPPORT\\i/e};
	\node at (VDR-i) {FREQUENCE\\DE PERCUSSION};
	\node at (VDR-F) {DEPIT\\PULSE};
	\node at (VDR-O) {CPAP\\OSCILLANTE};
	\node at (VDR-I) {TEMPS\\INSPIRATOIRE};
	\node at (VDR-E) {TEMPS\\EXPIRATOIRE};
	\end{scope}

	\begin{scope}[
		every path/.style={
			black,
			opacity=0.80,
			line width=0.7mm,
			->
			},
			]
	\draw [] (1) to (2);
	\draw [bend left=27] (2) to (3);
	\draw [bend left=60] (3) to (4);
	\draw [bend left=45] (4) to (5);
	\draw [bend left=45] (5) to (6);
	\end{scope}

\end{tikzpicture}

	\caption{Séquence de réglage des paramètres.}
\end{figure*}

En raison des interactions entre les différents réglages, il est
judicieux de régler en premier les paramètres ayant beaucoup d'influence
sur les autres réglages, ou influençant plusieurs autres réglages.

Ainsi, avant d'effectuer quelque réglage que ce soit, on s'assurera que
la pression de travail est réglée à 40 lb/po\textsuperscript{2} et que la nébulisation est
en fonction. On s'assurera aussi que la PEP non oscillante et
l'augmentation des pressions de convection (3\textsuperscript{e} phase) sont désactivées
(tourné complètement en sens horaire).

Ensuite, étant donné que le rapport \ie\ des percussions (haute
fréquence) influence à la fois la fréquence de percussion et l'amplitude
des percussions (donc les pressions de ventilation), il est judicieux
d'ajuster ce paramètre en tout premier lieu.

Une fois le rapport \ie\ des percussions ajusté,~le temps inspiratoire
des percussions peut être ajusté à n'importe quel moment pour régler la
fréquence de percussion.

Pour les paramètres d'amplitude de percussion, l'amplitude des
percussions pendant l'inspiration influence celle pendant l'expiration.
Il convient donc de toujours ajuster la pression inspiratoire avant la
pression expiratoire.

Finalement, les pressions de ventilation ayant une influence sur le
temps inspiratoire et expiratoire de la convection (basse fréquence),
on attendra d'avoir ajusté les pressions de ventilation avant de régler
avec précision ces deux paramètres.
