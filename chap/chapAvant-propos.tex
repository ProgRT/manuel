\chapter{Avant-propos}
Énormément de choses restent à découvrir au sujet du VDR-4 et de la ventilation
qu'il délivre. Son utilité clinique reste, pour plusieurs, encore à démontrer.
Et quant à ses mécanismes d'action et la façon optimale de l'utiliser, c'est
l'absence quasi complète de données.

Néanmoins, de la même façon que le jazz est une musique de musiciens, le VDR-4 est
un ventilateur pour les passionnés de ventilation mécanique.  Si c'est votre
cas, vous ne pourrez pas rester insensible à la singularité de cet appareil et
de son mode ventilation original.

Malgré mon absence quasi totale d'expérience clinique avec cet appareil (hé oui!),
j'ose formuler le souhait que cet ouvrage contribue, ne serait-ce qu'un tant soit peut, à combler l'énorme
manque de matériel pédagogique de qualité à son sujet\footnote{À commencer par la {\em déplorable} documentation du manufacturier.} et soit utile à ceux qui, comme moi,
cherchent à comprendre ce que c'est que cette bibitte là.

Bonne lecture!

\cleardoublepage
\tableofcontents
