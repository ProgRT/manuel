\chapter{Composantes du système}

\section{Module de contrôle}

Le module de contrôle est la composante qui permet de régler les paramètres de
la ventilation délivrée par le VDR-4.  À partir de son alimentation en gaz à
haute pression (air et oxygène), le module de contrôle produit :

\begin{itemize}
	\item Un débit intermittent alimentant le phasitron (connecteur et tubulure blanche),
	\item Un débit continue (+/- 20 l/min) alimentant le nébuliseur ou tout autre système d’humidification (si activé) (connecteur et tubulure jaune),
	\item Un débit auxiliaire (+/- 10 l/min) (ajouté à la sortie du nébuliseur dans le circuit classique) (connecteur et tubulure verte).
\end{itemize}

Il est aussi doté d’un port de monitorage (connecteur et tubulure rouge). 
Un multimètre numérique affiche les pressions moyennes (inspiratoire,
expiratoire et globale) et les fréquences (percussion et convection).
Finalement, le module de contrôle comprend aussi une alarme de déconnection
alimentée par une pile.

Le fonctionnement du module de contrôle est exclusivement pneumatique, à
l’exception du multimètre et de l’alarme de déconnection.  Chaque bouton
actionné par l’utilisateur est une valve contrôlant une cartouche pneumatique. 

Le circuit logique du module de contrôle est constitué d’un agencement d’une
trentaine de cartouches pneumatiques.  Cette conception a pour résultat que
plusieurs paramètres réglables s’inter-influencent.  Par exemple, une
augmentation de l’amplitude des percussions à l’inspiration (bouton DEBIT PULSE)
entrainera aussi une augmentation  de l’amplitude des percussions à
l’expiration.

%\begin{figure}
%	\caption{Panneau avant du module de contrôle.}
%\end{figure}

\begin{figure}[h]
	\begin{minipage}{0.49\textwidth}
	\centering
		\vspace{.25cm}
	\begin{tikzpicture}
		\pic [name=C] {cartouche};
		\pic {obturateur-bas};

		\draw [double distance=5.3mm,]
		(CO) -- ++(2,0) (CO) ++ (1,0) |- (CS)
		node [pos=0.25] {$\downarrow$}
		node [pos=0.75] {$\leftarrow$}
		;

		\draw (CI) node {$\downarrow$};
		\draw (CO) node {$\rightarrow$};

	\end{tikzpicture}
	A - Cartouche ouverte

\end{minipage}
\begin{minipage}{0.49\textwidth}
	\centering
		\vspace{.25cm}
	\begin{tikzpicture}

		\pic [name=C] {cartouche};
		\pic {obturateur-haut};

		\draw [double distance=5.3mm,]
		(CO) -- ++(2,0) (CO) ++ (1,0) |- (CS) 
		;

		\draw (CI) node {$\downarrow$};

	\end{tikzpicture}
	B - Cartouche fermée

\end{minipage}

	\caption{Fonctionnement d'une cartouche pneumatique. À mesure que la pression
	augmente derrière le diaphragme, celui-ci se déforme, emmenant  le piston à
	obstruer l’arrivée de gaz.} 
\end{figure}

\section{Phasitron}
\index{phasitron}

Le phasitron est la composante du circuit de ventilation raccordée directement à
l’interface patient (tube endotrachéal, canule de trachéotomie, etc.). Il
remplit les deux fonctions suivantes : 

\begin{itemize} 
	\item Amplification du jet de gaz (percussion) en provenance du module de contrôle,
	\item Valve expiratoire.
\end{itemize}

L’amplification du jet de gaz se fait par un appel d’air (principe de venturi).
Le ratio air aspiré : air injecté du tube de venturi diminue au fur et à mesure
que la pression augmente à la sortie de celui-ci. Conçu en tant que mécanisme de
protection pulmonaire, cette caractéristique tend à diminuer l’amplitude des
variations de pressions de ventilation lors de changement de mécanique
pulmonaire.

Lorsqu’un débit d’air est injecté dans le tube de venturi, il se déplace vers
l’avant du phasitron, obstruant ainsi l’orifice expiratoire (voir
Figure \ref{fig:phasitron-coupe}). Lorsque le tube de venturi ne reçoit plus de
débit, il retourne à sa position de repos (à l’arrière du phasitron), libérant
ainsi l’orifice expiratoire.

L’absence de circuit respiratoire entre le phasitron et l’interface patient
ainsi que l’utilisation de tubulures peu compliantes entre le phasitron et le
module de contrôle évitent l’atténuation des percussions dans le volume
compressible du circuit.

\begin{figure}
	\begin{tikzpicture}[
		every node/.style={
			align=center
		}
]
\pic [name=Phasitron, scale=0.5] {phasitron};
	\node [left=2mm] at (Phasitron-Pt) {Patient $\leftrightarrow$};
	\node [below=2mm] at (Phasitron-E) {$\downarrow$\\Expiration};
	\node [below=2mm, align=center] at (Phasitron-A) {$\uparrow$\\Appel d'air};
	\node [right=2mm] at (Phasitron-S) {$\leftarrow$ Pressurisation};
	\node [below=2mm, anchor=north east] at (Phasitron-M) {Monitorage};
\end{tikzpicture}

	\caption{Le phasitron.}
\end{figure}

\begin{figure}
	\def\pScale{0.5}
\centering
\begin{minipage}{.45\textwidth}
	\centering
	\tikzsetnextfilename{fig-phasitron_inspi}
	\begin{tikzpicture}[
			scale=\pScale,
			every node/.style={transform shape}
			]

			\pic [name=P, draw=black!50, fill=gray!10] {phasitron-coupe};
			\pic {venturi-avance};
			\path (P-S) -- (P-Pt) 
			node [pos=0.32] (J) {}
			coordinate [pos=0.15] (D)
			;


			\draw [line width=.2mm, ->] (P-S) ++(3mm,0) to (D);
			\draw [
				line width=.2mm,
				->, 
				shorten <=1mm,
				] (D) to (J);

				\draw [
					line width=.5mm, 
					->, 
					out=90, 
					in=-45,
					shorten >=1mm
					] (P-A) ++ (0, -3mm)  to (J);

					\draw [line width=1mm, ->] (J)  to (P-Pt);

	\end{tikzpicture}

	A - Inspiration
\end{minipage}\hfill
\begin{minipage}{0.45\textwidth}
	\centering

	\tikzsetnextfilename{fig-phasitron_expi}
		\begin{tikzpicture}[
				scale=\pScale,
				every node/.style={transform shape}
				]

				\pic [name=P, draw=black!50, fill=gray!10] {phasitron-coupe};
				\pic {venturi-recule};

				\draw [
					line width=1mm, 
					->, 
					out=0, 
					in=90, 
					looseness=1.8,
					] ([yshift=-4mm]P-Pt) to (P-E);

		\end{tikzpicture}

		B - Expiration
\end{minipage}

	\caption{Fonctionnement du phasitron. Le débit en provenance du module de
contrôle déforme le diaphragme et déplace le tube de venturi vers l’avant lors
de l’inspiration, obstruant ainsi l’orifice expiratoire. À l’expiration, le
diaphragme reprend sa forme initiale et ramène le tube de venturi vers
l’arrière, libérant ainsi l’orifice expiratoire.} 
	\label{fig:phasitron-coupe}
\end{figure}

\section{Système d’humidification}

Le système d’humidification de base du VDR-4 est un nébuliseur pneumatique.
Celui-ci est utilisé pour humidifier les gaz qui sont aspirés par le tube de
venturi du phasitron. Le circuit d’humidification est conçu de façon à :

\begin{itemize}
	\item S’assurer qu’un débit suffisant est disponible à l’orifice d’appel d’air du phasitron,
	\item Évacuer le débit excédentaire,
	\item Permettre au patient de respirer facilement l’air ambiant en cas de défaillance de l’appareil.
\end{itemize}

Plusieurs institutions utilisant le VDR-4 jugent ce système 
d’humidification insuffisant et le combine ou le remplace par un 
(ou même deux) humidificateur chauffant (voir Figure \ref{fig:circuit}).

%\begin{figure}
%	\caption{Circuit d'humidification "classique" du VDR-4.}
%\end{figure}

\begin{figure}
	\tikzsetnextfilename{fig-circuit}
\begin{tikzpicture}[
	scale=0.32,
	very thin,
]

\begin{scope}[every node/.style={ transform shape }]
	\pic [name=F] at(0,0) {fp};
	\pic [name=N] at(15,10) {neb};
	\pic [yscale=1.22, xscale=-1.22, name=P] at (25, 35) {phasitron};
	\pic [scale=1.8, name=VDR] at (-20,40) {vdr};
	\pic [name=M] at(N-CR) {manifold};
	\pic  at (M-BC) {bag};
\end{scope}

%%%%%%%%%%%%%%%%%%%%%%%%%%%%%%%%%%%%%%%%%%%%%%%%%%
% Hight presure  circuit
%%%%%%%%%%%%%%%%%%%%%%%%%%%%%%%%%%%%%%%%%%%%%%%%%%

\begin{scope}[every path/.style={double distance=0.8mm}]
	\draw (VDR-C1)  -| (P-M);
	\draw (VDR-C2) -- ++(8,0) |- (P-S);
	\draw (VDR-C4) -- ++(6, 0) -- ++(0, -30) -| (N-CN);
\end{scope}

\begin{scope}[every node/.style={transform shape}]
	\pic [yshift=11mm, xscale=-1] (fsa) at (P-A) {failsafeValve};
	\pic [yshift=11mm] (fse) at (P-E) {failsafeValve};
\end{scope}

%%%%%%%%%%%%%%%%%%%%%%%%%%%%%%%%%%%%%%%%%%%%%%%%%%
% Humidification circuit
%%%%%%%%%%%%%%%%%%%%%%%%%%%%%%%%%%%%%%%%%%%%%%%%%%

\begin{scope}[
	every path/.style={
		double distance=7mm,
		looseness=2,
		shorten <=-1mm,
		shorten >=-1mm
	}
]

% From the exalation to the manifold
\draw (fse-CB) to [out=-90, in=0, sloped, sloped, near start]  node {$\rightarrow$} (M-RC);

% From the nebuliser to the FishePaykel
\draw [double=black!13](N-CL) to [out=180, in=90, sloped, near start]  node {$\leftarrow$} (F-CR);

% From the FisherPaykel to the phasitron
\draw [double=black!13](F-CL) to [out=90, in=270, sloped] node {$\rightarrow$} (fsa-CB);

\end{scope}

% valves direction
\node foreach \pos in {a, e} [xscale=-1, font=\tiny] at(fs\pos-V) {$\mapsto$};

\end{tikzpicture}

	\label{fig:circuit}
	\caption{Intégration d'un humidificateur chauffant au circuit du VDR-4.}
\end{figure}

\section{Module de monitorage (Monitron)}

Le Monitron est un moniteur électronique complètement indépendant du module de
contrôle. Il vise à étendre les capacités de monitorage limitées de celui-ci.

Le signal de pression est transmis du module de contrôle au Monitron au moyen
d’une tubulure se trouvant dans l’espace entre les deux appareils.

\subsection{Données monitorées}

Les données numériques fournies par le Monitron sont les suivantes :

\begin{itemize}
\item Pression de crête inspiratoire,
\item Pression de crête expiratoire,
\item Pression moyenne,
\item Temps inspiratoire (convection),
\item Temps expiratoire (convection),
\item Fréquence (convection),
\item Ratio I:E (convection),
\item Fréquence (percussion),
\item Ratio i:e (percussion),
\item Heure.
\end{itemize}

\subsection{Alarmes}

Une alarme de basse pression et une alarme de haute pression peuvent être ajustées.

L’alarme de haute pression se déclenche dès que la pression lue est supérieure
au seuil d’alarme réglé.

L’alarme de pression basse se déclenche lorsque la pression lue est inférieure
au seuil d’alarme réglé pour une durée supérieure à 30 secondes.

La touche SET ajuste automatiquement l’alarme basse à 2 cmH₂O et l’alarme haute
à 10 cmH₂O au-dessus de la pression de crête inspiratoire.
